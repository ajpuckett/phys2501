% ****** Start of file template.aps ****** %
%%
%%
%%   This file is part of the APS files in the REVTeX 4 distribution.
%%   Version 4.0 of REVTeX, August 2001
%%
%%
%%   Copyright (c) 2001 The American Physical Society.
%%
%%   See the REVTeX 4 README file for restrictions and more information.
%%
%
% This is a template for producing manuscripts for use with REVTEX 4.0
% Copy this file to another name and then work on that file.
% That way, you always have this original template file to use.
%
% Group addresses by affiliation; use superscriptaddress for long
% author lists, or if there are many overlapping affiliations.
% For Phys. Rev. appearance, change preprint to twocolumn.
% Choose pra, prb, prc, prd, pre, prl, prstab, or rmp for journal
%  Add 'draft' option to mark overfull boxes with black boxes
%  Add 'showpacs' option to make PACS codes appear
%  Add 'showkeys' option to make keywords appear
\documentclass{revtex4}
%\documentclass[aps,prl,preprint,superscriptaddress]{revtex4}
%\documentclass[aps,prl,twocolumn,groupedaddress]{revtex4}
\usepackage[dvipdf]{graphicx}
%\usepackage{dcolumn}

% You should use BibTeX and apsrev.bst for references
% Choosing a journal automatically selects the correct APS
% BibTeX style file (bst file), so only uncomment the line
% below if necessary.
%\bibliographystyle{apsrev}

\begin{document}

% Use the \preprint command to place your local institutional report
% number in the upper righthand corner of the title page in preprint mode.
% Multiple \preprint commands are allowed.
% Use the 'preprintnumbers' class option to override journal defaults
% to display numbers if necessary
%\preprint{}

%Title of paper
\title{Quick Reference Sheet for Linear Fits}

% repeat the \author .. \affiliation  etc. as needed
% \email, \thanks, \homepage, \altaffiliation all apply to the current
% author. Explanatory text should go in the []'s, actual e-mail
% address or url should go in the {}'s for \email and \homepage.
% Please use the appropriate macro foreach each type of information

% \affiliation command applies to all authors since the last
% \affiliation command. The \affiliation command should follow the
% other information
% \affiliation can be followed by \email, \homepage, \thanks as well.
%\author{Physics 2501: Mechanics and Electromagnetism Laboratory}
\author{R.T. Jones}
%\homepage[]{Your web page}
%\thanks{}
%\altaffiliation{}
\affiliation{Dept. of Physics, University of Connecticut}
%\author{R.T. Jones}
%\affiliation{University of Connecticut}

%Collaboration name if desired (requires use of superscriptaddress
%option in \documentclass). \noaffiliation is required (may also be
%used with the \author command).
%\collaboration can be followed by \email, \homepage, \thanks as well.
%\collaboration{}
%\noaffiliation

\date{\today}

%\begin{abstract}
% insert abstract here
%\end{abstract}

% insert suggested PACS numbers in braces on next line
%\pacs{}
% insert suggested keywords - APS authors don't need to do this
%\keywords{}

\setlength{\topmargin}{0in}

%\maketitle must follow title, authors, abstract, \pacs, and \keywords
\maketitle

% body of paper here - Use proper section commands
% References should be done using the \cite, \ref, and \label commands

%% The normal text is displayed in two-column format, but special
%% sections spanning both columns can be inserted within the page
%% format so that long equations can be displayed. Use
%% sparingly.
%%\begin{widetext}
%% put long equation here
%%\end{widetext}
%
%% figures should be put into the text as floats.
%% Use the graphics or graphicx packages (distributed with LaTeX2e)
%% and the \includegraphics macro defined in those packages.
%% See the LaTeX Graphics Companion by Michel Goosens, Sebastian Rahtz,
%% and Frank Mittelbach for instance.
%%
%% Here is an example of the general form of a figure:
%% Fill in the caption in the braces of the \caption{} command. Put the label
%% that you will use with \ref{} command in the braces of the \label{} command.
%% Use the figure* environment if the figure should span across the
%% entire page. There is no need to do explicit centering.
%
%%\begin{turnpage}
%% Surround figure environment with turnpage environment for landscape
%% figure
%% \begin{turnpage}
%% \begin{figure}
%% \includegraphics{}%
%% \caption{\label{}}
%% \end{figure}
%% \end{turnpage}
%
%% tables should appear as floats within the text
%%
%% Here is an example of the general form of a table:
%% Fill in the caption in the braces of the \caption{} command. Put the label
%% that you will use with \ref{} command in the braces of the \label{} command.
%% Insert the column specifiers (l, r, c, d, etc.) in the empty braces of the
%% \begin{tabular}{} command.
%% The ruledtabular enviroment adds doubled rules to table and sets a
%% reasonable default table settings.
%% Use the table* environment to get a full-width table in two-column
%% Add \usepackage{longtable} and the longtable (or longtable*}
%% environment for nicely formatted long tables. Or use the the [H]
%% placement option to break a long table (with less control than 
%% in longtable).
%
%
%% Surround table environment with turnpage environment for landscape
%% table
%% \begin{turnpage}
%% \begin{table}
%% \caption{\label{}}
%% \begin{ruledtabular}
%% \begin{tabular}{}
%% \end{tabular}
%% \end{ruledtabular}
%% \end{table}
%% \end{turnpage}
%
%% Specify following sections are appendices. Use \appendix* if there
%% only one appendix.
%%\appendix
%%\section{}
%

Graphing calculators and spreadsheet applications typically have
built-in functionality for superimposing a trend line on a 
two-dimensional set of data points.  Use of such crude tools is not
appropriate for the physics experiments in this course.  Consider an
experiment in which you make repeated measurements of a quantity denoted
$y$ which depends upon one independent variable $x$ that you are able to
control.  You choose a series of values $x_i$, $i=1\ldots n$ and measure
the corresponding values $y_i$ whose measurement RMS errors are $\sigma_i$.
Furthermore, let us suppose for simplicity that the theoretical prediction
$y(x)$ for how $y$ should depend on $x$ is a straight line.
\[
y(x) = mx + b
\]
The values of $m$ and $b$ which provide the ``best fit'' (minimum value
of the $\chi^2$) are given for this simple case by the following formulas.
\begin{equation}
m = \frac{S_{00}S_{11}-S_{10}S_{01}}{S_{20}S_{00}-S_{10}^2}
\end{equation}
\begin{equation}
b = \frac{S_{20}S_{01}-S_{10}S_{11}}{S_{20}S_{00}-S_{10}^2}
\end{equation}
where the $S$ symbols are shorthand for different sums over the data
values, defined as follows.
\begin{equation}
S_{pq} = \sum_{i=1}^{n}\frac{x_i^py_i^q}{(\Delta y_i)^2}
\end{equation}
where the integers $p$ and $q$ can be 0, 1, or 2.
The associated uncertainties on the best-fit values for $m$ and $b$ are 
\begin{eqnarray}
\Delta m &=& \sqrt{\frac{S_{00}}{S_{20}S_{00}-S_{10}^2}} \\
\Delta b &=& \sqrt{\frac{S_{20}}{S_{20}S_{00}-S_{10}^2}}
\end{eqnarray}

\end{document}
%%
%% ****** End of file template.aps ******
%
%
