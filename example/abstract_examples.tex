\documentclass{article}
\usepackage{amsmath}
\begin{document}
\section{Abstract Example 1: Nature 466, 213-216 (08 July 2010)}

The proton is the primary building block of the visible Universe, but many
of its properties-such as its charge radius and its anomalous magnetic 
moment are not well understood. The root-mean-square charge radius, r$_p$,
has been determined with an accuracy of 2 per cent (at best) by electron-proton
scattering experiments. The present most accurate value of r$_p$ (with an
uncertainty of 1 per cent) is given by the CODATA compilation of physical
constants. This value is based mainly on precision spectroscopy of atomic
hydrogen and calculations of bound-state quantum electrodynamics (QED).
The accuracy of r$_p$ as deduced from electron-proton scattering limits
the testing of bound-state QED in atomic hydrogen as well as the
determination of the Rydberg constant (currently the most accurately measured
fundamental physical constant). An attractive means to improve the accuracy
in the measurement of r$_p$ is provided by muonic hydrogen (a proton
orbited by a negative muon); its much smaller Bohr radius compared to
ordinary atomic hydrogen causes enhancement of effects related to the
finite size of the proton. In particular, the Lamb shift (the energy
difference between the 2S1/2 and 2P1/2 states) is affected by as much
as 2 per cent. Here we use pulsed laser spectroscopy to measure a muonic
Lamb shift of 49,881.88(76) GHz. On the basis of present calculations of
fine and hyperfine splittings and QED terms, we find r$_p$ = 0.84184(67)fm,
which differs by 5.0 standard deviations from the CODATA value of 
0.8768(69)fm. Our result implies that either the Rydberg constant has to
be shifted by 110kHz/c (4.9 standard deviations), or the calculations of
the QED effects in atomic hydrogen or muonic hydrogen atoms are insufficient.

\section{Abstract Example 2: Physical Review Letters 108, 112502 (2012)}

We report the first measurement of the parity-violating asymmetry A$_{PV}$
in the elastic scattering of polarized electrons from $^{208}$Pb. A$_{PV}$
is sensitive to the radius of the neutron distribution (R$_n$). The result
A$_{PV} = 0.656 \pm 0.060$ (stat) $\pm 0.014 $(syst) ppm corresponds to a 
difference between the radii of the neutron and proton distributions 
$R_n - R_p = 0.33 +0.16 -0.18$ fm and provides the first electroweak
observation of the neutron skin which is expected in a heavy, neutron-rich nucleus.

\section{Abstract Example 3: Physical Review Letters 104, 242301 (2010)}
Among the most fundamental observables of nucleon structure, electromagnetic
form factors are a crucial benchmark for modern calculations describing the
strong interaction dynamics of the nucleon's quark constituents; indeed,
recent proton data have attracted intense theoretical interest. In this
Letter, we report new measurements of the proton electromagnetic form factor
ratio using the recoil polarization method, at momentum transfers 
Q$^2$=5.2, 6.7, and 8.5GeV$^2$. By extending the range of Q$^2$ for which
$G_E^p$ is accurately determined by more than 50\%, these measurements will
provide significant constraints on models of nucleon structure in the
nonperturbative regime.

\section{Abstract Example 4: Physics Letters B 716 (2012) 1-29}
A search for the Standard Model Higgs boson in proton-proton collisions
with the ATLAS detector at the LHC is presented. The datasets used correspond
to integrated luminosities of approximately 4.8 fb$^-1$ collected at
$\sqrt{s} = 7$ TeV in 2011 and 5.8 fb$^-1$ at $\sqrt{s} = 8$ TeV in 2012.
Individual searches in the channels H->ZZ$^(*)$->llll, H->gamma gamma and
H->WW->e nu mu nu in the 8 TeV data are combined with previously published
results of searches for H->ZZ$^(*)$, WW$^(*)$, bbbar and tau$^+$tau$^-$ in
the 7 TeV data and results from improved analyses of the H->ZZ$^(*)$->llll
and H->gamma gamma channels in the 7 TeV data. Clear evidence for the
production of a neutral boson with a measured mass of 126.0 $\pm$ 0.4(stat)
$\pm$ 0.4(sys) GeV is presented. This observation, which has a significance
of 5.9 standard deviations, corresponding to a background fluctuation
probability of 1.7x10$^{-9}$, is compatible with the production and decay
of the Standard Model Higgs boson.

\section{Abstract Example 5: Phys.Lett. B 716 (2012) 30-61}
Results are presented from searches for the standard model Higgs boson in
proton-proton collisions at sqrt(s) = 7 and 8 TeV in the Compact Muon
Solenoid experiment at the LHC, using data samples corresponding to 
integrated luminosities of up to 5.1 inverse femtobarns at 7 TeV and 
5.3 inverse femtobarns at 8 TeV. The search is performed in five decay 
modes: gamma gamma, ZZ, WW, tau tau, and b b-bar. An excess of events 
is observed above the expected background, with a local significance of
5.0 standard deviations, at a mass near 125 GeV, signalling the production
of a new particle. The expected significance for a standard model Higgs 
boson of that mass is 5.8 standard deviations. The excess is most 
significant in the two decay modes with the best mass resolution, 
gamma gamma and ZZ; a fit to these signals gives a mass of 
125.3 $\pm$ 0.4 (stat.) $\pm$ 0.5 (syst.) GeV. The decay to two photons 
indicates that the new particle is a boson with spin different from one.

\end{document}
