\documentclass{revtex4}
\usepackage{amsmath}
\usepackage{graphicx}
\begin{document}
\title{Lecture notes on probability and statistics relevant to experiments in PHYS 2501W}
\author{Prof. Andrew Puckett}
\affiliation{Dept. of Physics, University of Connecticut}
\date{\today}
\maketitle
\section{Definition of Probability Distribution, Mean, and Variance}
For a continuous random variable $x$, we define the probability density function or distribution $P(x)$ such that $\int_a^b P(x) dx$ is the probability that $x$ lies between $a$ and $b$. $P(x)$ is normalized so that its integral over all possible values of $x$ is one:
\begin{eqnarray}
  \int_{-\infty}^{\infty} P(x) dx = 1
\end{eqnarray}
We define the \emph{expectation value} of a function or operator $\mathcal{O}(x)$ as 
\begin{eqnarray}
  \left<\mathcal{O}(x) \right> &=& \int_{-\infty}^{\infty} \mathcal{O}(x) P(x) dx
\end{eqnarray}
We define the mean $\mu$ and the variance $\sigma^2$ of $x$ as 
\begin{eqnarray}
  \mu &=& \left<x\right> = \int_{-\infty}^\infty x P(x) dx \nonumber \\
  \sigma^2 &=& \left<(x-\mu)^2\right> = \int_{-\infty}^{\infty} (x - \mu)^2 P(x) dx \nonumber \\
  &=& \int_{-\infty}^{\infty} (x^2 - 2\mu x + \mu^2) P(x) dx = \left<x^2\right> - \left<x\right>^2
\end{eqnarray}
\section{Sampling statistics}
In experimental physics, we will often perform repeated measurements
of a quantity that is expected to have the same true value, in order
to obtain an estimate of the uncertainty in the quantity, and to
reduce the uncertainty in that quantity by averaging several
independent measurements. In any individual measurement, the observed
value $x_{obs}$ will differ from the \emph{unknown} ``true'' value $x_t$
by an unknown amount $d_x = x_{obs} - x_t$. We can think of $x_{obs}$
as a random variable sampled from a probability distribution with mean
$\mu = x_t$ and standard deviation $\sigma$ characterizing the typical
error in a single measurement. Assuming
that our measurements are \emph{unbiased}, the theoretical mean of the
distribution of $x_{obs}$ is $\lim_{N \rightarrow \infty}
\left<\frac{1}{N} \sum_{i=1}^N x_i \right> = \mu = x_t$. In other words, we assume that we are
equally likely to err in either direction. Assume we have sampled the distribution of $x$ values by
performing $N$ measurements $x_i$ for $i=1,\ldots,N$. We define the \emph{sample mean} $\bar{x}$ and the \emph{sample variance} $s^2$ as 
\begin{eqnarray}
  \bar{x} &=& \frac{1}{N} \sum_{i=1}^{N} x_i \label{samplemean} \\
  s^2 &=& \frac{1}{N} \sum_{i=1}^{N} (x_i - \bar{x})^2 = \frac{1}{N} \sum_{i=1}^N x_i^2 -\frac{2}{N} \sum_{i=1}^{N} x_i \bar{x} + \frac{1}{N} \sum_{i=1}^N \bar{x}^2 \nonumber \\
  s^2 &=& \overline{x^2} - \bar{x}^2 \label{samplevariance},
\end{eqnarray}
where on the last line we have used the fact that
$\sum_{i=1}^N \bar{x}^2 =  N\bar{x}^2$ and
$\frac{-2}{N}\sum_{i=1}^N x_i \bar{x} = -2\bar{x}^2$. It follows from
the assumption of unbiased measurements that the expectation value of
the sample mean is simply the ``true'' value $x_t = \mu$. Second, let us
consider the variance of the sample mean. Since the expectation value
of the mean (the ``average of the average'') is $\mu$, the variance of
the sample mean is simply the average squared deviation of the mean of a sample of
size $N$ from the ``true'' value $\mu$:
\begin{eqnarray}
  \sigma_{\bar{x}}^2 &=& \left<(\bar{x} - \mu)^2 \right>
\end{eqnarray}
Let $d_i = x_i - \mu$ be the ``true error'' in a single measurement,
and let $D = \bar{x} - \mu$ be the ``true error'' in the sample mean. Then
$d_i - D = x_i - \bar{x}$ by definition.

The variance of the sample mean is evaluated by averaging $D^2$ over
the distribution of $x$:
\begin{eqnarray}
  \sigma^2_{\bar{x}} &=& \left<D^2\right> \nonumber \\
  &=& \left<\left(\frac{1}{N} \sum_{i=1}^N x_i - \mu\right)^2\right>
  \nonumber \\
  &=& \frac{1}{N^2}\left<\left(\sum_{i=1}^N x_i - N\mu \right)\left(\sum_{j=1}^N
      x_j - N\mu \right)\right> \nonumber \\
  &=& \frac{1}{N^2} \left<\left(\sum_{i=1}^N (x_i - \mu) \right)\left(\sum_{j=1}^N
      (x_j - \mu) \right) \right>\nonumber \\
  \sigma^2_{\bar{x}} &=& \frac{1}{N^2} \left<\sum_{i=1}^N d_i^2 +
    \sum_{i=1}^N \sum_{j \ne i}^N d_i d_j \right>  = \frac{\sigma^2}{N} \label{varianceofsamplemean},
\end{eqnarray}
where on the last line, we have replaced $(x_i - \mu)$ with $d_i$. Consider what happens when we take the expectation value in
Eq.~\eqref{varianceofsamplemean}. The expectation value of the ``true
error'' $d_i$ is zero by the assumption of unbiased
measurements. The second term inside the $\left<\ldots \right>$ in
Eq.~\eqref{varianceofsamplemean} vanishes because each of the $N$
measurements in each sample is assumed to be independent; i.e., the
error $d_i$ is uncorrelated with the error $d_j$ for $i \ne
j$. When we average this term over the distribution of a large number
of $N$-samplings, we get zero. Therefore, we are left with the term $\frac{1}{N^2}
\left<\sum_{i=1}^N d_i^2 \right>$. The expectation value
$\left<\sum_{i=1}^N d_i^2\right>$ just gives $N \left<d^2\right> = N
\sigma^2$, because each sample gives $N$ random $d^2$ values, and
the average of $d^2$ over the distribution is just $\sigma^2$, by
definition. We have thus arrived at the famous result that the
variance of the sample mean equals the variance of the parent distribution
divided by the number of measurements in the sample. Thus, if the
standard deviation $\sigma$ of the distribution of $x$ serves as a
measure of the error in a single measurement, then $\sigma_{\bar{x}}
= \sigma/\sqrt{N}$ serves as a measure of the error in the average
of $N$ independent measurements of the same quantity.

Let us consider next the relationship between the sample variance and
the distribution variance. The
expectation value of the sample variance is given by
\begin{eqnarray}
  \left<s^2 \right> &=& \frac{1}{N} \left<\sum_{i=1}^N (x_i -
    \bar{x})^2 \right> = \frac{1}{N} \left<\sum_{i=1}^N (d_i -D)^2
  \right> \nonumber \\
  &=& \frac{1}{N} \left<\sum_{i=1}^N \left( d_i^2 + D^2 - 2d_i D
    \right)\right> \nonumber \\
  &=& \sigma^2 + \frac{\sigma^2}{N} - \frac{2}{N} \left<\sum_{i=1}^N
    (x_i-\mu)(\bar{x}-\mu)\right> = \sigma^2 + \frac{\sigma^2}{N} -
  \frac{2}{N} \left<N (\bar{x}-\mu)^2 \right> \nonumber \\
  &=& \sigma^2 + \frac{\sigma^2}{N} - \frac{2}{N}\left<N D^2 \right> =
  \sigma^2 - \frac{\sigma^2}{N} = \sigma^2 \frac{N-1}{N} 
\end{eqnarray}
We see that the sample variance is a \emph{biased} estimator for the
variance of the distribution, and tends to underestimate the variance
at small values of $N$. As $N$ becomes large, however, the factor
$(N-1)/N$ multiplying $\sigma^2$ tends toward one. Moreover, we see
that $s^2 \frac{N}{N-1} = \frac{1}{N-1}\sum_{i=1}^N (x_i - \bar{x})^2$
is an \emph{unbiased} estimator for $\sigma^2$.
\section{Special Probability Distributions: Binomial, Poisson and
  Gaussian}
\subsection{Binomial}
The binomial distribution describes the probability distribution of outcomes of
binary (success/failure) experiments. Specifically, the probability
$P(k)$ of observing $k$ ``successes'' in $N$ independent trials when the
probability of success is $p$ is given by:
\begin{eqnarray}
  P(k) &=& \frac{N!}{k!\left(N-k\right)!} p^k (1-p)^{N-k} \label{binomialdist}
\end{eqnarray}
The various factors appearing in Eq.~\eqref{binomialdist} are
intuitively simple to understand. The factor $p^k$ represents the product of the
individual probabilities of each of the $k$ successes. The factor
$(1-p)^{N-k}$ represents the product of the individual probabilities
for each of the $N-k$ failures. However, we don't care about the order
in which the successes and failures occur, so we have to multiply this
probability by the number of possible ways of choosing $k$ successes
from $N$ trials. It is a well-known fact that the number of possible
orderings of $N$ trials, regardless of success or failure, is
$N!$. Let $M$ be the number of possible orderings of $k$ successes
drawn from among $N$ trials. There are $N$ ways to choose the first
(successful) trial. For each possible choice of the first trial, there
are $N-1$ ways to choose the second trial, and so on all the way up to $N - k + 1$ ways to choose
the $k^{th}$ trial: 
\begin{eqnarray}
  M &=& N (N-1) (N-2) \ldots (N-k + 1) = N (N-1) (N-2) \ldots (N-k +
  1) \times \frac{(N-k)!}{(N-k)!} = \frac{N!}{(N-k)!}
\end{eqnarray}
Recalling that we don't care about the ordering of the $k$ successes,
the number of ways of choosing $k$ successes from $N$ trials is simply
given by $M/k! = \frac{N!}{k!(N-k)!}$.

The mean of the binomial distribution is given by 
\begin{eqnarray}
  <k> &=& \sum_{k=0}^N k P(k) \nonumber \\
  &=& \sum_{k=0}^N k \frac{N!}{k! (N-k)!} p^k (1-p)^{N-k} \nonumber \\
  &=& \sum_{k=0}^N (N-k+1) \frac{p}{1-p} \frac{N!}{(k-1)! (N-k+1)!} p^{k-1}(1-p)^{N-k+1} \nonumber \\
  &=& \frac{p}{1-p} \sum_{k=0}^N (N-k+1) P(k-1) = \frac{p}{1-p} \left[
\end{eqnarray}

\end{document}
